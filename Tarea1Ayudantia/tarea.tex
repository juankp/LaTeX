\documentclass[10pt,a4paper]{article}
\usepackage[latin1]{inputenc}
\usepackage[spanish]{babel}
\begin{document}

\begin{center}
{Tarea 1}
{Metodolog\'ia de desarrollo de software}
\end{center}


\section{Metodolog\'ia Tradicional}


 CMMI  es un modelo para la mejora y evaluaci\'on de procesos para el desarrollo mantenimiento y operaci\'on de sistemas de software En la actualidad hay dos \'areas de inter\'es cubiertas por los modelos de CMMI: Desarrollo y Adquisici\'on
 

La versi\'on actual de CMMI es la versi\'on 12 Existen, a su vez tres sabores dentro de la versi\'on disponible:


CMMI para Desarrollo (CMMI-DEV o CMMI for Development), que trata  sobre los procesos de desarrollo de productos y servicios de software.

CMMI para la adquisici\'on (CMMI-ACQ o CMMI for Acquisition), que trata sobre la gesti\'on de la cadena de suministro, adquisici\'on y contrataci\'on externa en los procesos del gobierno y la industria.

CMMI para servicios CMMISVC o CMMI for Services actualmente en borrador est\'a disenado para cubrir todas las actividades que requieren gestionar establecer y entregar Servicios
Dentro del sabor CMMI-DEV, existen dos modelos:

CMMI-DEV CMMIDEV IPPD Integrated Product and Process Development
Las pr\'acticas de CMMI deben adaptarse a cada organizaci\'on en funci\'on de sus objetivos de negocio

Las organizaciones no pueden ser certificadas CMMI Por el contrario, una organizaci\'on es evaluada usando un m\'etodo de evaluacivon conocido como SCAMPI y recibe una calificaci\'on (o acreditaci\'on) de nivel 2 a 5 si sigue los niveles de madurez se comienza ya estando en el nivel 1

Tambi\'en es posible que la organizaci\'on seleccione \'areas de proceso y en vez de por niveles de madurez puede obtener los niveles de capacidad en cada una de las \'areas de Proceso obteniendo el "Perfil de Capacidad" de la Organizaci\'on

El modelo CMMI v12 CMMIDEV contiene las siguientes 22 \'areas de proceso:



\end{document}