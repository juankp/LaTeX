\section{Metodolog\'{i}a \'{A}gil: Scrum}


El desarrollo \'{A}gil es una respuesta metodol\'{o}gica a las dificultades del actual desarrollo de Software, con el fin de organizar de manera efectiva y lograr un mejor resultado del software proporcionado.

Las Metodolog\'{i}as \'{A}giles se basan en un desarrollo iterativo e incremental, teniendo en cuenta un orden, por el cual los requerimientos y las soluciones progresan, por lo que disminuye el tiempo entre iteraciones.

En cada iteraci\'{o}n debe existir: Una planificaci\'{o}n, An\'{a}lisis de Requerimientos, Dise\~{n}o, Codificaci\'{o}n, Revisi\'{o}n y Documentaci\'{o}n. Al t\'{e}rmino de cada iteraci\'{o}n se debe reevaluar las precedencias del proyecto.\\

Scrum es una metodolog\'{i}a \'{A}gil. En un tipo de referencia, que tiene una serie de pr\'{a}cticas y roles, del cual se puede partir, para definir el proceso que se llevar\'{a} a cabo durante el proyecto. Los roles principales en Scrum son el ScrumMaster, que trabaja de forma similar al Jefe de proyecto, el ProductOwner, que representa a los stakeholders (interesados externos o internos), y el Team que incluye a los desarrolladores. 

Existen varias implementaciones de sistemas para gestionar el proceso de Scrum, que van desde notas amarillas post-it y pizarras hasta paquetes de software.
Una de las mayores ventajas de Scrum es que es muy f\'{a}cil de aprender, y requiere muy poco esfuerzo para comenzarse a utilizar.\\\

En pocas palabras, Scrum:

\begin{enumerate}
\item Divide su organizaci\'{o}n en peque\~{n}os equipos interfuncionales y autoorganizados.

\item Divide su trabajo en una lista de productos peque\~{n}os y concretos. Ordenar la lista por orden de prioridad y estimar el esfuerzo relativo de cada elemento.

\item Divide el tiempo en corto de longitud fija iteraciones (normalmente 1-4 semanas), con c\'{o}digo potencialmente entregable demostrado despu\'{e}s de cada iteraci\'{o}n.

\item Sobre la base de los conocimientos adquiridos mediante la inspecci\'{o}n de la liberaci\'{o}n despu\'{e}s de cada iteraci\'{o}n, optimiza el plan de lanzamiento y actualizaci\'{o}n de las prioridades de la colaboraci\'{o}n con el cliente.

\item Optimiza el proceso por el que tiene una retrospectiva despu\'{e}s de cada iteraci\'{o}n.
\end{enumerate}
