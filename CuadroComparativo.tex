\documentclass[10pt,a4paper]{article}
\usepackage[latin1]{inputenc}
\usepackage[spanish]{babel}
\usepackage[latin1]{inputenc}
\usepackage[T1]{fontenc}
\begin{document}

\section{Tabla Comparativa}
\begin{table}[h!]
\centering
\begin{tabular} {|p{6cm}|p{6cm}|} \hline
Metodologias Agiles & Metodologias Tradicionales \\\hline
Basadas en heur\'isticas provenientes de pr\'acticas de
producci\'on de c\'odigo & Basadas en normas provenientes de est\'andares
seguidos por el entorno de desarrollo\par \\\hline
Especialmente preparados para cambios durante el 
proyecto & Cierta resistencia a los cambios\\\hline
Impuestas internamente (por el equipo) & Impuestas externamente\\\hline
Proceso menos controlado, con pocos principios & Proceso mucho m\'as controlado, con numerosas
pol\'iticas/normas\\\hline
No existe contrato tradicional o al menos es bastante flexible & Existe un contrato prefijado\\\hline
El cliente es parte del equipo de desarrollo & El cliente interact\'ua con el equipo de desarrollo\\\hline
Grupos peque\~nos (menor a 10 integrantes) y trabajando en el mismo sitio & Grupos grandes y posiblemente distribuidos mediante reuniones\\\hline
Pocos artefactos & M\'as artefactos\\\hline
Pocos roles & M\'as roles\\\hline
Menos \'enfasis en la arquitectura del software & La arquitectura del software es esencial y se expresa mediante modelos\\\hline


\end{tabular}
\caption{Tabla comparativa metodolog\'ias \'agiles y tradicionales}
\end{table}

\end{document}